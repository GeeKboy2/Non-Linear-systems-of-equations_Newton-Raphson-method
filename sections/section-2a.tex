\section{Les points de Lagrange}
Le but de cette partie est de trouver le point d'équilibre d'un objet sachant qu'il est soumis à plusieurs forces différentes.

\subsection{Modélisation de différentes forces}
Il a donc fallu dans un premier écrire des algorithmes qui vont permettre de calculer la force appliquer à un point en fonction de ses
coordonées cartésiennes. Les différentes forces qui ont été choisies sont : la force élastique, la force centrifuge, et la force gravitationnelle.
Les calculs qui suivent ont demandés aussi de calculer leur Jacobienne et de les implémenter en Python. Ce travail de modélisation a donc demandé plus
de travail de calcul que de travail algorithmique.


Les forces sont représentées sous les formes suivantes :

\begin{equation*}
    f_e : \left[\begin{array}{cccccc}
        x \\
        y \\
      \end{array} \right]
      \rightarrow
      \left[\begin{array}{cccccc}
        k(x-x0) \\
        k(y-y0) \\
      \end{array} \right]
\end{equation*}
\begin{equation*}
    f_g : \left[\begin{array}{cccccc}
        x \\
        y \\
      \end{array} \right]
      \rightarrow
      \left[\begin{array}{cccccc}
        -k\cdot \frac{x-x0}{((x-x0)^2+(y-y0)^2)^{\frac{3}{2}}} \\
        -k\cdot \frac{y-y0}{((x-x0)^2+(y-y0)^2)^{\frac{3}{2}}} \\
      \end{array} \right]
\end{equation*}

\subsection{Recherche de points d'équilibre}
Dans un premier nous avons recherché les points d'équilibres 
